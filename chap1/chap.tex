\ifx\allfiles\undefined
\documentclass[12pt, a4paper, oneside, UTF8]{ctexbook}
\def\path{../config}
\input{../config/_config}
\begin{document}
% \input{../config/cover}
\else
\fi
%标题
\chapter{域}
	\section{域的定义}
		\subsection{域}
			\begin{defn}{域}{}
				设$F$是一个集合,如果存在两个运算$+:F \times F \rightarrow F$和$\cdot :F \times F \rightarrow F$,分别称为加法和乘法,并且满足:
				
				\ding{172}(加法单位元存在)存在一个元素$0_F \in F$,称为零元,$\forall x \in F,x+0_F=0_F+x=x$
				
				\ding{173}(加法逆存在)$ \forall x \in F,\exists (-x) \in F,\text{s.t.}x+(-x)=(-x)+x=0_F$,$(-x)$称为$x$的加法逆元
				
				\ding{174}(加法交换律)$\forall x,y \in F,x+y = y+x$
				
				\ding{175}(加法结合律)$\forall x,y,z \in F,(x+y)+z = x+(y+z)$
				
				\ding{176}(乘法单位元存在)存在一个元素$1_F \in F,1_F \neq 0_F$,称为一元,$\forall x \in F,x\cdot 1_F=1_F \cdot x = x$
				
				\ding{177}(乘法逆存在)$\forall x \in F-{0_F},\exists x^{-1} \in F,\text{s.t. }x\cdot x^{-1}=x^{-1} \cdot x = 1$,$x^{-1}$称为$x$的乘法逆元
				
				\ding{178}(乘法交换律)$\forall x,y \in F,x\cdot y = y\cdot x$
				
				\ding{179}(乘法结合律)$\forall x,y,z \in F,(x\cdot y)\cdot z = x\cdot (y\cdot z)$
				
				\ding{180}(乘法分配律)$\forall x,y,z \in F,x\cdot(y+z)=x\cdot y+x\cdot z$
			\end{defn}
		\subsection{域的性质}
		\begin{para}{0}
			\point{}
				\begin{proposition}
					加法和乘法的单位元是唯一的。
				\end{proposition}
				\begin{proof}
					先考虑加法的单位元。假设命题不成立,那么我们不妨假设$0_1,0_2$都是$F$的零元,$0_1 \neq 0_2$
					
					那么$0_1=0_1+0_2=0_2$,于是有$0_1 = 0_2$,与假设矛盾。于是加法的单位元唯一。
					
					同理可证,乘法的单位元也是唯一的。
				\end{proof}
			\point{}
				\begin{proposition}
					$\forall a$,加法的逆$-a$是唯一的。
					
					如果还有$a \neq 0$,那么乘法的逆$a^{-1}$也是唯一的。
				\end{proposition}
				\begin{proof}
					先考虑加法逆,不妨假设命题不成立,那么$\exists b,c,a+b=0,a+c=0,b \neq c$
					
					于是,$b = b+ 0=b+(a+c)=(a+b)+c=0+c=c$,这与假设矛盾。于是加法逆唯一。
					
					同理可证,乘法逆也是唯一的。
				\end{proof}
			\point{}
				\begin{proof}
					$a\cdot 0 = 0$
				\end{proof}
				\begin{proof}
					$a \cdot 0=a\cdot (0+0)=a\cdot 0+a\cdot 0$
					
					$\Rightarrow a\cdot 0+(-a \cdot a)=a\cdot 0 + a\cdot 0+(-a\cdot 0)$
					
					$\Rightarrow 0=a \cdot 0$
				\end{proof}
				立即有以下推论:
				\begin{corollary}{}{}
					$ab=0 \Rightarrow a = 0 \vee b=0$
				\end{corollary}
				\begin{proof}
					假设$a \neq 0$,那么$b = a^-1 \cdot 0=0$,命题得证
				\end{proof}
			\point{}
				\begin{proposition}
					$-a = (-1)\cdot a$
				\end{proposition}
				\begin{proof}
					$a + (-a) = 0 = \left(1+ (-1)\right) \cdot a = 1 \cdot a + (-1)\cdot a=a+(-1)\cdot a$
					
					$\Rightarrow (-a)+a+(-a)=(-a)+a+(-1)\cdot a$
					
					$\Rightarrow -a = (-1)\cdot a$
				\end{proof}
				随后我们即可得出以下推论
				\begin{corollary}{}{}
					$(-1)\cdot (-x) = x$
				\end{corollary}
				\begin{proof}
					我们只需证明:$(-1)(-1)=1$
					
					因为$(-1)(-1)+(-1)\cdot 1 = 0$
					
					$\Rightarrow (-1)(-1) + (-1) = 0 \Rightarrow (-1)(-1)=1$
					
					那么,$(-1)(-x)=(-1)(-1)\cdot x = 1\cdot x = x$
				\end{proof}
				\begin{corollary}{}{}
					$(-x)(-x)=x \cdot x$
				\end{corollary}
				\begin{proof}
					运用前面的推论中的结果,$(-x)(-x)=x\cdot (-1)(-1)\cdot x = x\cdot 1\cdot x = x\cdot x$
				\end{proof}
		\end{para}
			
	\section{域的同态}
		\begin{defn}{域的同态}
			设$F_1,F_2$是两个域,如果存在一个映射$\varphi:F_1 \rightarrow F_2$,满足:
			
			\ding{172} $\varphi (0_{F_1}) = 0_{F_2}$
			
			\ding{173} $\varphi (1_{F_1}) = 1_{F_2}$
			
			\ding{174} $\varphi (x+y) = \varphi (x)+\varphi (y)$
			
			\ding{175} $\varphi (x\cdot y) = \varphi (x) \cdot \varphi (y)$
		\end{defn}
		值得注意的是,与我们之前了解到的线性空间同构不同,域的同态完全没有对映射的满射性、单射性作任何限制。但是,
		
		以下定理证明,两个域如果同态,那么同态映射是一个单射
		\begin{them}{域同态的单射性}{}
			若$\varphi : F_1 \rightarrow F_2$是$F_1$到$F_2$的同态,那么$\varphi$是单射
		\end{them}
		\begin{proof}
			不妨假设命题不成立。于是,$\exists x_1 \neq x_2 \text{s.t. }\varphi(x_1)=\varphi(x_2)$
			
			那么有:$\varphi (x_1-x_2)=\varphi (x_1)-\varphi (x_2) = 0_{F_2}$
			
			因为我们已经假设了$x_1 \neq x_2$,于是$(x_1-x_2)^{-1}$存在。将上式乘以$\varphi\left((x_1-x_2)^{-1}\right)$得:
			
			$1_{F_2} = \varphi (1_{F_1}) \varphi\left((x_1-x_2)^{-1} (x_1-x_2)\right) = \varphi\left((x_1-x_2)^{-1}\right)\varphi\left((x_1-x_2)\right)=0_{F_1}$
			
			与$0_{F_2} \neq 1_{F_2}$矛盾,于是命题得证。
		\end{proof}
		在证明这一点后,我们可以类似地引入域的同构:
		\begin{defn}{域的同构}{}
			设$\varphi : F_1 \rightarrow F_2$是$F_1$到$F_2$的同态
			
			如果$\varphi$还是个满射,那么我们称$\varphi$是一个同构;
			
			特别地,如果有$F_1=F_2$,我们称$\varphi$是一个自同构。
		\end{defn}
		并且引入自同构的不动域的概念:
		\begin{defn}{自同构域的不动域}{}
			设$\sigma : F \rightarrow F$是$F$的自同构,那么我们称集合
			
			$\{x \in F | \sigma(x) = x\}$为$F$的不动域
		\end{defn}
		“不动域”这一名称是合理的,因为利用域同构的定义容易证明不动域是一个域,而且是$F$的一个子域。
	\section{域的特征}
		\subsection{域的特征的定义}
			\begin{defn}{域的特征}{}
				设$F$是一个域,定义以下映射$N:\N \ni n \mapsto n_F \in F$,满足:
				
				$N(0)=0_F,N(n+1)=n_F+1_F$
				
				那么,如果$N$是一个单射,我们称$F$的特征为$0$,记作$\text{Char} F=0$;
				
				否则,我们将满足$N(p)=0_F,p > 0$的最小正整数称为$F$的特征,记作$\text{Char} F = p$。
			\end{defn}
			我们首先需要证明的是,任何一个域都是具有特征的,因为对于定义中的第二种情形,我们并不知道这样的$p$是否一定存在。
			\begin{them}{域特征的存在性}{}
				任何域$F$的特征$\text{Char} F$均存在
			\end{them}
			\begin{proof}
				我们只需要证明第二种情形。
				
				容易证明,$N(m+n)=N(m)+N(n)$。(仿照Peano公理下证明加法性质的方式即可)
				
				于是,因为$N$不是单射,于是一定有$a,b\in \N ,a > b,N(a)=N(b)$
				
				于是有$N(a-b)=N(a)-N(b)=0_F$。
				
				那么$\{m|N(m)=0_F\}\neq \varnothing$,因此这样的最小整数$\text{Char}F$存在
			\end{proof}
		接下来考虑几个性质
		\subsection{域的特征的性质}
			\begin{proposition}
				设$F$是一个域,那么或者$\text{Char }F=0$,或者$\text{Char }F =p$是素数
			\end{proposition}
			\begin{proof}
				我们只需要证明当$\text{Char }F =p>0$时,$p$是素数

				不妨假设命题不成立,那么一定有$1 < q < p,1 < r < p,p=qr$

				容易证明,$N(qr)=N(q)N(r)$。(仿照Peano公理下证明乘法性质的方式即可)

				但是,因为$N(qr)=0$,于是$N(q)=0 \vee N(r) = 0$,这与定义中$p$是使$N(x)=0$成立的最小正整数矛盾。

				于是命题得证。
			\end{proof}
			\begin{them}{同态域的特征相等}{}
				设$\varphi : E \to F$是域$E$到域$F$的同态

				那么有:$\text{Char }E = \text{Char }F$
			\end{them}
			\begin{proof}
				我们不妨假设结论不成立。

				首先我们证明,不可能$\text{Char }E =0, \text{Char }F = p,p$为素数。

				此时,$\varphi \left(N_E(p)\right)=N_F(p)=0_F$,此处$N_E,N_F$分别是$N$在对应$E,F$的情况下的映射

				那么按照同态的定义,一定有$N_E(p)=0_E$,但是这与$\text{Char E}=0$矛盾。

				那么只需要考虑当$\text{Char }E =p, \text{Char }F=q,p,q$为素数,$p \neq q$

				那么,有$\varphi \left(N_E(q)\right)=N_F(q)=0_F$

				那么按照同态的定义,一定有$N_E(q)=0_E$,这说明$p \mid q$,这与$q$是素数矛盾。

				所以假设不成立,命题得证。
			\end{proof}
	\section{域的扩张}
		\subsection{域的扩张的定义}
			\begin{defn}{子域}{}
				设$E,F$是两个域,$E \subseteq F$
				
				如果$0_F,1_F \in E$,并且$F$中的加法和乘法对$E$形成一个域,
				
				那么我们称$E$是$F$的一个子域,并称$F$是$E$的一个域扩张,
				
				记作$F\textbackslash E$
			\end{defn}
			如果$E \textbackslash F$,$F \cong G$,那么我们也称:在同构意义下$E \textbackslash G$。

			此后我们所说的子域,默认指的是同构意义下的子域。

			借助域的扩张的概念,我们可以证明一些比较简单的结论
			\begin{proposition}
				域$F$如果有$\text{Char }F=0$,那么$\Q$是它的子域;如果有$\text{Char }F=p$,那么$F_p$是它的子域
			\end{proposition}
			\begin{proof}
				如果$\text{Char }F=0$,那么$F$是无限集。因为$0_F,1_F \in F$,那么一定也有$n_F \in F,n \in \N$。

				那么,$-n_F \in F$;进一步地,一定有$a \cdot b^{-1} \in F,a,b \in \{\pm n_F\}$。

				那么,一定有$F \textbackslash \Q$。

				如果$\text{Char }F=p$,$p$是素数。那么,$\{0_F,\cdots,(p-1)_F\}$由映射$N$的性质一定是一个域。

				那么,一定有$F \textbackslash F_p$
			\end{proof}
		\subsection{有限扩张}
			从域的定义容易看出,$F$也可以视为$E$上的一个线性空间。
			\begin{defn}{域的扩张次数}{}
				如果$F\textbackslash E$,那么我们记$[F:E]:=dim_E F$,并称$F$是$E$由$[F:E]$次扩张得到的。

				如果$[F:E]$有限,我们称$F$是$E$的有限扩张,反之称它是无限扩张。
			\end{defn}
			有限扩张的概念可以让我们立即得出以下结论
			\begin{them}{域的元素个数仅可能无限或者是$p^k$}{}
				一个域的元素个数,或者是无限,或者是$p^k$,其中$p$是一个素数,$k$是正整数
			\end{them}
			\begin{proof}
				事实上,我们只需要证明有限域$F$的个数只可能是$p^k$

				设$\text{Char }F = p$,那么一定有$F \textbackslash F_p$

				因为$F$是有限域,所以一定有$[F:F_p]=d$有限,那么此时$|F|=p^d$。因为$p$一定为素数,因此命题得证。
			\end{proof}
		\subsection{有限生成扩张}
			\begin{defn}{域的生成扩张}{}
				设$E,F$是两个域,$E\textbackslash F$,集合$S \subseteq E$

				那么我们定义包含$F,S$中全部元素的最小域,即

				$F(S) := \bigcap\limits_{(F \cup S) \subseteq K,E \textbackslash K} K$

				称为在$F$上由$S$生成的$E$的子域
			\end{defn}
			\begin{defn}{有限生成与无限生成}{}
				设$E,F$是两个域,$E\textbackslash F$

				如果存在一个集合$S \subseteq E,F(S) = E$,那么我们称$E$是$F$的有限生成扩张;

				如果对于任意的有限集$S \subseteq E,F(S) \neq E$,那么我们称$E$是$F$的无限生成扩张
			\end{defn}
			显然有以下性质
			\begin{proposition}
				有限扩张都是有限生成扩张
			\end{proposition}
			\begin{proof}
				设$E \textbackslash F,[E:F]=n < +\infty$。

				那么$E$是$F$上的一个线性空间,我们取它的一组基$\{\alpha_1,\cdots,\alpha_n\}$

				于是$E = \text{span}_F (\alpha_1,\cdots,\alpha_n)$。由线性生成的性质,那么有

				$E = F(\alpha_1,\cdots,\alpha_n)$。于是命题得证。
			\end{proof}
			值得注意的是,这个命题如果反过来则不成立,比如说:
			\begin{example}
				$\Q(e)$是$\Q$的有限生成扩张,但是不是$\Q$的有限扩张
			\end{example}
			\begin{proof}
				不妨假设命题不成立,那么有$[\Q(e):\Q]=n < +\infty$

				那么,因为线性空间中,数量多于维数的一组向量一定线性相关,那么$1,e,\cdots,e^n$线性相关
				
				$\Rightarrow \exists a_0,\cdots,a_n \in \Q,a_0+a_1 e +\cdots+a_n e^n=0$

				但是,这与$e$是超越数矛盾。于是命题得证
			\end{proof}
		\subsection{代数扩张}
			我们首先提出代数元的概念
			\begin{defn}{代数元}{}
				设$E \textbackslash F$是一个域扩张,$u \in E$如果满足:

				$\exists p(x) \in F[x],p(u)=0$,那么我们称$u$是$F$上的代数元
			\end{defn}
			我们如下定义代数扩张
			\begin{defn}{代数扩张和超越扩张}{}
				设$E \textbackslash F$是一个域扩张

				如果$\forall u \in E$,$u$是$F$的代数元,那么我们称$E$是$F$的代数扩张;

				反之,如果$\exists u \in E$,$u$不是$F$的代数元,那么我们称$E$是$F$的超越扩张。
			\end{defn}
			接下来我们考虑代数、有限、有限生成三种扩张之间的联系
			\begin{para}{0}
				\point{有限扩张都是代数扩张}
					\begin{them}{有限扩张都是代数扩张}{}
						任何有限扩张都是代数扩张
					\end{them}
					\begin{proof}
						设$E,F$是两个域,$[E:F]=n < +\infty$

						取$\forall \alpha \in E$,考虑集合$\{1,\alpha ,\cdots,\alpha^n\}$

						因为这个集合有$n+1 > [E:F]$个元素,因此它一定线性相关;这代表着

						$\exists a_0,\cdots,a_n$不全为$0,a_0 + a_1\alpha + \cdots a_n \alpha ^n=0$

						因此$\alpha $是多项式$\sum\limits_{k=0}^{n} a_k x^k \in F[x]$的一个根

						所以$\alpha $是$F$的代数元,于是命题得证。
					\end{proof}
				\point{有限扩张的组合和拆分也是有限的}
					\begin{proposition}
						设$E,K,F$是三个域,$E \textbackslash K,K\textbackslash F$,并且$E \textbackslash F$是有限扩张。

						那么,$E \textbackslash K,K\textbackslash F$是有限扩张,并且有$[E:F]=[E:K]\cdot [K:F]$
					\end{proposition}
					\begin{proof}
						我们首先证明$E \textbackslash K,K\textbackslash F$是有限扩张。

						显然,$K\textbackslash F$一定是有限扩张,因为$K$是$E$在$F$上的线性子空间,而$[E:F]$有限。

						我们不妨假设$E\textbackslash K$不是有限的,那么一定可以取一组无限基$\{e_\alpha | \alpha  \in A\}$

						考虑$F$上的线性组合$\sum\limits_{i=1}^{\infty} a_i e_{\alpha_i},\alpha_i \in A$

						我们可以断言:这个线性组合在$\exists a_i \neq 0$时不为零,因为$F \subseteq K$,所以这个线性组合也可以视为$K$上的,
						
						而线性无关向量组的子向量组也是线性无关的;

						但是,这是不可能的:因为我们知道$[K:F]$有限,一个无限集不可能线性无关。所以$[E:K]$一定有限。

						接下来证明$[E:F]=[E:K]\cdot [K:F]$

						假设$[E:K]=m,[K:F]=n$,取$E$在$K$上的一组基$\{\alpha_1,\cdots,\alpha_m\}$,$K$在$F$上的一组基$\{\beta_1,\cdots,\beta_n\}$

						$\forall \eta \in E,\exists a_1,\cdots,a_m,\eta =a_1 \alpha_1+\cdots+a_m \alpha_m$

						对系数作展开,有:

						$\exists b_{ij},\eta =(b_{11}\beta_1+\cdots+b_{1n}\beta_n) \alpha_1+\cdots+(b_{m1}\beta_1+\cdots+b_{mn}\beta_n) \alpha_m$

						$=\sum\limits_{i=1,j=1}^{i=m,j=n} b_{ij}\alpha_i \beta_j$

						$=\sum\limits_{i=1}^{m}\left(\sum\limits_{j=1}^{n}b_{ij} \beta_j\right) \alpha_i$

						令$\eta =0$,因为$\{\alpha_1,\cdots,\alpha_m\}$线性无关,一定有$\sum\limits_{j=1}^{n}b_{ij} \beta_j=0$

						但是,$\{\beta_1,\cdots,\beta_n\}$线性无关,所以一定有$b_{ij}=0$。所以$\{\alpha_i \beta_j\}$是$E$在$F$上的一组基

						所以$[E:F]=mn=[E:K]\cdot [K:F]$,命题得证
					\end{proof}
					利用这个命题可以得出以下推论
					\begin{corollary}{素数次扩张不存在平凡子域}{}
						若$E \textbackslash F$是一个有限扩张,$[E:F]=p$为素数

						那么$E,F$没有非平凡的中间域,即$E\textbackslash K\textbackslash F \Leftrightarrow E=K \vee  F=K$
					\end{corollary}
					\begin{proof}
						这是显然的,因为素数的因子仅有$1$和自身;而$[K:F]=1$当且仅当$K,F$在同构意义下相等。
					\end{proof}
					\begin{corollary}{单代数扩张一定是有限扩张}{}
						设$E,F$是两个域,$E=F(u),u$是$F$的代数元
						
						那么$E=F(u)$是一个有限扩张,并且$[E:F]=\text{deg }f(x)$,其中$f(x)$是$u$的极小多项
					\end{corollary}
					\begin{proof}
						取$u$的极小多项式$f(x) \in F[x]$,设$f(x)=\sum\limits_{k=0}^{n} a_k x^k,a_n =1$

						因为$f(u)=0$,所以有$x^n = -\sum\limits_{i=0}^{n-1} a_k x^k$

						于是,任意的$\alpha ^k$均可以由$1,\alpha,\cdot,\alpha^{n-1}$线性表出,这表示

						$E = F(u)=\{a_0 +\cdots+a_{n-1} u^{n-1}\}$

						于是$E$中任意一个元素可由$\{1,\cdots,u^{n-1}\}$线性表出。但是$u$的极小多项式的次数为$n$,所以这个集合一定线性无关,也就是它是$E$的基

						所以$[E:F]=n < +\infty$
					\end{proof}
				\point{有限生成的代数扩张是有限扩张}
					\begin{them}{有限生成的代数扩张是有限扩张}{}
						$E \textbackslash F$是有限扩张$\Leftrightarrow E=F(u_1,\cdots,u_n)$其中$u_1,\cdots,u_n$是$F$的代数元
					\end{them}
					\begin{proof}
						首先证明充分性。
						
						取$E$在$F$上的一组基$\{u_1,\cdots,u_n\}$

						我们说,一定有$E=F(u_1,\cdots,u_n)$,因为:

						$E=\text{span}_F (u_1,\cdots,u_n) \subseteq F(u_1,\cdots,u_n)$

						但是,同时也有$F(u_1,\cdots,u_n) \subseteq E$,因为$E$是一个域,而$u_1,\cdots,u_n \in E$

						所以只需要证明$u_1,\cdots,u_n$是$F$的代数元。但是,有限扩张都是代数扩张,按照代数扩张的定义可知这是成立的。

						接下来证明必要性。这是更加显然的,因为我们已经证明了单代数扩张有限,那么有

						$[F(u_1,\cdots,u_n):F]=[F(u_1,\cdots,u_{n}):F(u_1,\cdots,u_{n-1})]\cdots [F(u_1):F] < +\infty$
					\end{proof}
				\point{代数扩张的复合也是代数扩张}
					\begin{them}{代数扩张的复合也是代数扩张}{}
						设$E\textbackslash K\textbackslash F$,

						如果$E\textbackslash K,K\textbackslash F$都是代数扩张,那么$E\textbackslash F$也是代数扩张
					\end{them}
					\begin{proof}
						取$\forall \alpha \in E$
						
						因为$E\textbackslash K$是代数扩张,所以$\exists f(x) \in K[x],f(\alpha)=0$

						设$f(x)=a_0 + \cdots + a_n x^n$,取$R=F(a_0,\cdots,a_n)$

						因为$a_0,\cdots,a_n \in K$,而$K \textbackslash F$是代数扩张,所以$a_0,\cdots,a_n$是$F$的代数元

						那么,按照前面的定理,一定有$[R:F] < +\infty$

						接下来考虑$R(\alpha)$,因为$a_0,\cdots,a_n \in R$,所以$\alpha$是$R$的代数元,那么就有$[R(\alpha):R]< \infty$

						所以$[R(\alpha):F]=[R(\alpha):R]\cdot [R:F]<+\infty$

						但是,$R \textbackslash F$,所以一定也有$[F(\alpha):F] < +\infty$,又因为有限扩张都是代数扩张,所以$\alpha$是$F$的代数元,于是命题得证。
					\end{proof}
			\end{para}
		三种扩张的关系可以用以下图示概括
		\begin{tikzcd}
			\text{Infinite extension} \arrow[rr, "is", shift left] \arrow[dd, "is", shift left] &  & \text{Algebratic extension} \arrow[ll, "if infinite-generate", shift left] \\
																				 &  &                                       \\
			\text{Infinite-generate extension} \arrow[uu, "if Algebratic", shift left]                            &  &                                      
		\end{tikzcd}
	\section{代数闭包}
		\begin{defn}{相对代数闭包}{}
				设$E \textbackslash F$是一个域扩张,那么我们称
				
				$K  = \{\alpha \in E | \alpha \text{是}F\text{上的代数元}\}$是$F$在$E$上的相对代数闭包
		\end{defn}
		\begin{defn}{代数闭域}{}
			如果域$K$没有真代数扩张,即$K$的任意一个代数扩张$K^{\prime} \textbackslash K$都有$K^{\prime}=K$

			那么我们称$K$是一个代数闭域
		\end{defn}
		\begin{defn}{绝对代数闭包}{}
			如果$\overline{F} \textbackslash F$是一个代数扩张,且$\overline{F}$是一个代数闭域
			
			那么我们称$\overline{F}$是$F$的绝对代数闭包,记作$\overline{F}$
		\end{defn}

		我们早就发现,似乎代数扩张并不能无限的扩张,而是会有一个终点,这个终点其实就是代数闭包,以下命题指出了这个事实。

		\begin{proposition}
			如果$K$是$F$在$E$上的代数闭包,
			
			那么如果$E \textbackslash K^{\prime} \textbackslash K$且$K^{\prime} \textbackslash K$是代数扩张,那么$K^{\prime}=K$
		\end{proposition}
		\begin{proof}
			因为$K \textbackslash F,K^{\prime} \textbackslash K$都是代数扩张,所以$K^{\prime} \textbackslash F$也是代数扩张

			因为代数闭包即是全部可以通过代数扩张得到的元素的集合,所以必定有$K^{\prime} = K$
		\end{proof}
		显然绝对代数闭包的定义和以下等价
		\begin{proposition}
			如果$K$是$F$在$E$上的相对代数闭包,且$E$是一个代数闭域,那么$K$是$F$的绝对代数闭包
		\end{proposition}
		\begin{proof}
			只需证明$K$是一个代数闭域

			没法证,得用商环…………
		\end{proof}
		\begin{them}{绝对代数闭包的存在性}{}
			任意一个域$F$的绝对代数闭包$\overline{F}$都存在,并且在同构意义下唯一
		\end{them}
		\begin{proof}
			要用Zorn引理……
		\end{proof}
\ifx\allfiles\undefined
\end{document}
\fi