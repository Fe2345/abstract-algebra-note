\ifx\allfiles\undefined
\documentclass[12pt, a4paper, oneside, UTF8]{ctexbook}
\def\path{../config}
\input{../config/_config}
\begin{document}
% \input{../config/cover}
\else
\fi
%标题
\chapter{域}
	\section{域的定义}
		\begin{defn}{域}{}
			设$F$是一个集合,如果存在两个运算$+:F \times F \rightarrow F$和$\cdot :F \times F \rightarrow F$,分别称为加法和乘法,并且满足:
			
			\ding{172}(加法单位元存在)存在一个元素$0_F \in F$,称为零元,$\forall x \in F,x+0_F=0_F+x=x$
			
			\ding{173}(加法逆存在)$ \forall x \in F,\exists (-x) \in F,\text{s.t.}x+(-x)=(-x)+x=0_F$,$(-x)$称为$x$的加法逆元
			
			\ding{174}(加法交换律)$\forall x,y \in F,x+y = y+x$
			
			\ding{175}(加法结合律)$\forall x,y,z \in F,(x+y)+z = x+(y+z)$
			
			\ding{176}(乘法单位元存在)存在一个元素$1_F \in F,1_F \neq 0_F$,称为一元,$\forall x \in F,x\cdot 1_F=1_F \cdot x = x$
			
			\ding{177}(乘法逆存在)$\forall x \in F-{0_F},\exists x^-1 \in F,\text{s.t. }x\cdot x^-1=x^-1 \cdot x = 1$,$x^-1$称为$x$的乘法逆元
			
			\ding{178}(乘法交换律)$\forall x,y \in F,x\cdot y = y\cdot x$
			
			\ding{179}(乘法结合律)$\forall x,y,z \in F,(x\cdot y)\cdot z = x\cdot (y\cdot z)$
			
			\ding{180}(乘法分配律)$\forall x,y,z \in F,x\cdot(y+z)=x\cdot y+x\cdot z$
		\end{defn}
\ifx\allfiles\undefined
\end{document}
\fi