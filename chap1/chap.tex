\ifx\allfiles\undefined
\documentclass[12pt, a4paper, oneside, UTF8]{ctexbook}
\def\path{../config}
\usepackage{amsmath}
\usepackage{amsthm}
\usepackage{amssymb}
\usepackage{graphicx}
\usepackage{mathrsfs}
\usepackage{enumitem}
\usepackage{geometry}
\usepackage[colorlinks, linkcolor=black]{hyperref}
\usepackage{stackengine}
\usepackage{yhmath}
\usepackage{extarrows}
\usepackage{unicode-math}
\usepackage{tikz}
\usepackage{tikz-cd}
\usepackage{pifont}

\usepackage{fancyhdr}
\usepackage[dvipsnames, svgnames]{xcolor}
\usepackage{listings}

\definecolor{mygreen}{rgb}{0,0.6,0}
\definecolor{mygray}{rgb}{0.5,0.5,0.5}
\definecolor{mymauve}{rgb}{0.58,0,0.82}

\graphicspath{ {figure/},{../figure/}, {config/}, {../config/} }

\linespread{1.6}

\geometry{
    top=25.4mm, 
    bottom=25.4mm, 
    left=20mm, 
    right=20mm, 
    headheight=2.17cm, 
    headsep=4mm, 
    footskip=12mm
}

\setenumerate[1]{itemsep=5pt,partopsep=0pt,parsep=\parskip,topsep=5pt}
\setitemize[1]{itemsep=5pt,partopsep=0pt,parsep=\parskip,topsep=5pt}
\setdescription{itemsep=5pt,partopsep=0pt,parsep=\parskip,topsep=5pt}

\lstset{
    language=Mathematica,
    basicstyle=\tt,
    breaklines=true,
    keywordstyle=\bfseries\color{NavyBlue}, 
    emphstyle=\bfseries\color{Rhodamine},
    commentstyle=\itshape\color{black!50!white}, 
    stringstyle=\bfseries\color{PineGreen!90!black},
    columns=flexible,
    numbers=left,
    numberstyle=\footnotesize,
    frame=tb,
    breakatwhitespace=false,
} 
\usepackage[strict]{changepage} 
\usepackage{framed}
\usepackage{tcolorbox}
\tcbuselibrary{most}

\definecolor{greenshade}{rgb}{0.90,1,0.92}
\definecolor{redshade}{rgb}{1.00,0.88,0.88}
\definecolor{brownshade}{rgb}{0.99,0.95,0.9}
\definecolor{lilacshade}{rgb}{0.95,0.93,0.98}
\definecolor{orangeshade}{rgb}{1.00,0.88,0.82}
\definecolor{lightblueshade}{rgb}{0.8,0.92,1}
\definecolor{purple}{rgb}{0.81,0.85,1}

% #### 将 config.tex 中的定理环境的对应部分替换为如下内容
% 定义单独编号,其他四个共用一个编号计数 这里只列举了五种,其他可类似定义(未定义的使用原来的也可)
\newtcbtheorem[number within=section]{defn}%
{定义}{colback=OliveGreen!10,colframe=Green!70,fonttitle=\bfseries}{def}

\newtcbtheorem[number within=section]{lemma}%
{引理}{colback=Salmon!20,colframe=Salmon!90!Black,fonttitle=\bfseries}{lem}

% 使用另一个计数器 use counter from=lemma
\newtcbtheorem[use counter from=lemma, number within=section]{them}%
{定理}{colback=SeaGreen!10!CornflowerBlue!10,colframe=RoyalPurple!55!Aquamarine!100!,fonttitle=\bfseries}{them}

\newtcbtheorem[use counter from=lemma, number within=section]{criterion}%
{准则}{colback=green!5,colframe=green!35!black,fonttitle=\bfseries}{cri}

\newtcbtheorem[use counter from=lemma, number within=section]{corollary}%
{推论}{colback=Emerald!10,colframe=cyan!40!black,fonttitle=\bfseries}{cor}
% colback=red!5,colframe=red!75!black

% 这个颜色我不喜欢
%\newtcbtheorem[number within=section]{proposition}%
%{命题}{colback=red!5,colframe=red!75!black,fonttitle=\bfseries}{cor}

% .... 命题 例 注 证明 解 使用之前的就可以(全文都是这种框框就很丑了),也可以按照上述定义 ...
\renewenvironment{proof}{\par\textbf{证明:}\;}{\qed\par}
\newenvironment{solution}{\par{\textbf{解:}}\;}{\qed\par}
\newtheorem{proposition}{\indent 命题}[section]
\newtheorem{example}{\indent \color{SeaGreen}{例}}[section] % 绿色文字的 例 ,不需要就去除\color{SeaGreen}{}
\newtheorem*{rmk}{\indent 注}
\usepackage{amssymb}
\setmathfont{LatinModernMath-Regular}
\setmathfont[range=\mathbb]{TeXGyrePagellaMath-Regular}
\def\d{\mathrm{d}}
\def\R{\mathbb{R}}
\def\C{\mathbb{C}}
\def\Q{\mathbb{Q}}
\def\N{\mathbb{N}}
\def\Z{\mathbb{Z}}
\newcommand{\bs}[1]{\boldsymbol{#1}}
\newcommand{\ora}[1]{\overrightarrow{#1}}
\newcommand{\myspace}[1]{\par\vspace{#1\baselineskip}}
\newcommand{\xrowht}[2][0]{\addstackgap[.5\dimexpr#2\relax]{\vphantom{#1}}}
\newenvironment{ca}[1][1]{\linespread{#1} \selectfont \begin{cases}}{\end{cases}}
\newenvironment{vx}[1][1]{\linespread{#1} \selectfont \begin{vmatrix}}{\end{vmatrix}}
\newcommand{\tabincell}[2]{\begin{tabular}{@{}#1@{}}#2\end{tabular}}
\newcommand{\pll}{\kern 0.56em/\kern -0.8em /\kern 0.56em}
\newcommand{\dive}[1][F]{\mathrm{div}\;\bs{#1}}
\newcommand{\rotn}[1][A]{\mathrm{rot}\;\bs{#1}}
\usepackage{xeCJK}
\setCJKmainfont{SimSun}[BoldFont={SimHei}, ItalicFont={KaiTi}] % 设置中文支持

\newcommand{\point}[1]{\item {#1}}
\newenvironment{para}[1]{%
\ifcase#1\relax
\begin{enumerate}[label=\arabic*.] % 1.2.3.
\or
\begin{enumerate}[label=\textcircled{\arabic*}] % ①②③
\or
\begin{enumerate}[label=(\roman*)] % (i)(ii)(iii)
\else
\begin{enumerate}[label=\arabic*.] % 默认格式
\fi
}{
\end{enumerate}
}

\def\myIndex{0}
% \input{\path/cover_package_\myIndex.tex}

\def\myTitle{抽象代数笔记}
\def\myAuthor{Zhang Liang}
\def\myDateCover{\today}
\def\myDateForeword{\today}
\def\myForeword{前言标题}
\def\myForewordText{
    前言内容
}
\def\mySubheading{副标题}


\begin{document}
% \input{\path/cover_text_\myIndex.tex}

\newpage
\thispagestyle{empty}
\begin{center}
    \Huge\textbf{\myForeword}
\end{center}
\myForewordText
\begin{flushright}
    \begin{tabular}{c}
        \myDateForeword
    \end{tabular}
\end{flushright}

\newpage
\pagestyle{plain}
\setcounter{page}{1}
\pagenumbering{Roman}
\tableofcontents

\newpage
\pagenumbering{arabic}
\setcounter{chapter}{0}
\setcounter{page}{0}

\pagestyle{fancy}
\fancyfoot[C]{\thepage}
\renewcommand{\headrulewidth}{0.4pt}
\renewcommand{\footrulewidth}{0pt}








\else
\fi
%标题
\chapter{域}
	\section{域的定义}
		\subsection{域和子域}
			\begin{defn}{域}{}
				设$F$是一个集合,如果存在两个运算$+:F \times F \rightarrow F$和$\cdot :F \times F \rightarrow F$,分别称为加法和乘法,并且满足:
				
				\ding{172}(加法单位元存在)存在一个元素$0_F \in F$,称为零元,$\forall x \in F,x+0_F=0_F+x=x$
				
				\ding{173}(加法逆存在)$ \forall x \in F,\exists (-x) \in F,\text{s.t.}x+(-x)=(-x)+x=0_F$,$(-x)$称为$x$的加法逆元
				
				\ding{174}(加法交换律)$\forall x,y \in F,x+y = y+x$
				
				\ding{175}(加法结合律)$\forall x,y,z \in F,(x+y)+z = x+(y+z)$
				
				\ding{176}(乘法单位元存在)存在一个元素$1_F \in F,1_F \neq 0_F$,称为一元,$\forall x \in F,x\cdot 1_F=1_F \cdot x = x$
				
				\ding{177}(乘法逆存在)$\forall x \in F-{0_F},\exists x^-1 \in F,\text{s.t. }x\cdot x^-1=x^-1 \cdot x = 1$,$x^-1$称为$x$的乘法逆元
				
				\ding{178}(乘法交换律)$\forall x,y \in F,x\cdot y = y\cdot x$
				
				\ding{179}(乘法结合律)$\forall x,y,z \in F,(x\cdot y)\cdot z = x\cdot (y\cdot z)$
				
				\ding{180}(乘法分配律)$\forall x,y,z \in F,x\cdot(y+z)=x\cdot y+x\cdot z$
			\end{defn}
			我们再定义子域
			\begin{defn}{子域}{}
				设$E,F$是两个域,$E \subseteq F$
				
				如果$0_F,1_F \in E$,并且$F$中的加法和乘法对$E$形成一个域,
				
				那么我们称$E$是$F$的一个子域,并称$F$是$E$的一个域扩张,
				
				记作$F\textbackslash E$
			\end{defn}
			从以上定义容易看出,$F$也可以视为$E$上的一个线性空间。
			\begin{defn}{域的$n$次扩张}{}
				如果$F\textbackslash E$,那么我们记$[F:E]:=dim_E F$,并称$F$是$E$由$[F:E]$次扩张得到的。
			\end{defn}
	\section{域的同态}
		\begin{defn}{域的同态}
			设$F_1,F_2$是两个域,如果存在一个映射$\varphi:F_1 \rightarrow F_2$,满足:
			
			\ding{172} $\varphi (0_{F_1}) = 0_{F_2}$
			
			\ding{173} $\varphi (1_{F_1}) = 1_{F_2}$
			
			\ding{174} $\varphi (x+y) = \varphi (x)+\varphi (y)$
			
			\ding{175} $\varphi (x\cdot y) = \varphi (x) \cdot \varphi (y)$
		\end{defn}
		值得注意的是,与我们之前了解到的线性空间同构不同,域的同态完全没有对映射的满射性、单射性作任何限制。但是,
		
		以下定理证明,两个域如果同态,那么同态映射是一个单射
		\begin{them}{域同态的单射性}{}
			若$\varphi : F_1 \rightarrow F_2$是$F_1$到$F_2$的同态,那么$\varphi$是单射
		\end{them}
		\begin{proof}
			不妨假设命题不成立。于是,$\exists x_1 \neq x_2 \text{s.t. }\varphi(x_1)=\varphi(x_2)$
			
			那么有:$\varphi (x_1-x_2)=\varphi (x_1)-\varphi (x_2) = 0_{F_2}$
			
			因为我们已经假设了$x_1 \neq x_2$,于是$(x_1-x_2)^-1$存在。将上式乘以$\varphi\left((x_1-x_2)^-1\right)$得:
			
			$1_{F_2} = \varphi (1_{F_1}) \varphi\left((x_1-x_2)^-1 (x_1-x_2)\right) = \varphi\left((x_1-x_2)^-1\right)\varphi\left((x_1-x_2)\right)=0_{F_1}$
			
			与$0_{F_2} \neq 1_{F_2}$矛盾,于是命题得证。
		\end{proof}
		在证明这一点后,我们可以类似地引入域的同构:
		\begin{defn}{域的同构}{}
			设$\varphi : F_1 \rightarrow F_2$是$F_1$到$F_2$的同态
			
			如果$\varphi$还是个满射,那么我们称$\varphi$是一个同构;
			
			特别地,如果有$F_1=F_2$,我们称$\varphi$是一个自同构。
		\end{defn}
		并且引入自同构的不动域的概念:
		\begin{defn}{自同构域的不动域}{}
			设$\sigma : F \rightarrow F$是$F$的自同构,那么我们称集合
			
			$\{x \in F | \sigma(x) = x\}$为$F$的不动域
		\end{defn}
		“不动域”这一名称是合理的,因为利用域同构的定义容易证明不动域是一个域,而且是$F$的一个子域。
\ifx\allfiles\undefined
\end{document}
\fi