\ifx\allfiles\undefined
\documentclass[12pt, a4paper, oneside, UTF8]{ctexbook}
\setCJKmainfont{SimSun}
\def\path{../config}
\usepackage{amsmath}
\usepackage{amsthm}
\usepackage{amssymb}
\usepackage{graphicx}
\usepackage{mathrsfs}
\usepackage{enumitem}
\usepackage{geometry}
\usepackage[colorlinks, linkcolor=black]{hyperref}
\usepackage{stackengine}
\usepackage{yhmath}
\usepackage{extarrows}
\usepackage{unicode-math}
\usepackage{tikz}
\usepackage{tikz-cd}
\usepackage{pifont}

\usepackage{fancyhdr}
\usepackage[dvipsnames, svgnames]{xcolor}
\usepackage{listings}

\definecolor{mygreen}{rgb}{0,0.6,0}
\definecolor{mygray}{rgb}{0.5,0.5,0.5}
\definecolor{mymauve}{rgb}{0.58,0,0.82}

\graphicspath{ {figure/},{../figure/}, {config/}, {../config/} }

\linespread{1.6}

\geometry{
    top=25.4mm, 
    bottom=25.4mm, 
    left=20mm, 
    right=20mm, 
    headheight=2.17cm, 
    headsep=4mm, 
    footskip=12mm
}

\setenumerate[1]{itemsep=5pt,partopsep=0pt,parsep=\parskip,topsep=5pt}
\setitemize[1]{itemsep=5pt,partopsep=0pt,parsep=\parskip,topsep=5pt}
\setdescription{itemsep=5pt,partopsep=0pt,parsep=\parskip,topsep=5pt}

\lstset{
    language=Mathematica,
    basicstyle=\tt,
    breaklines=true,
    keywordstyle=\bfseries\color{NavyBlue}, 
    emphstyle=\bfseries\color{Rhodamine},
    commentstyle=\itshape\color{black!50!white}, 
    stringstyle=\bfseries\color{PineGreen!90!black},
    columns=flexible,
    numbers=left,
    numberstyle=\footnotesize,
    frame=tb,
    breakatwhitespace=false,
} 
\usepackage[strict]{changepage} 
\usepackage{framed}
\usepackage{tcolorbox}
\tcbuselibrary{most}

\definecolor{greenshade}{rgb}{0.90,1,0.92}
\definecolor{redshade}{rgb}{1.00,0.88,0.88}
\definecolor{brownshade}{rgb}{0.99,0.95,0.9}
\definecolor{lilacshade}{rgb}{0.95,0.93,0.98}
\definecolor{orangeshade}{rgb}{1.00,0.88,0.82}
\definecolor{lightblueshade}{rgb}{0.8,0.92,1}
\definecolor{purple}{rgb}{0.81,0.85,1}

% #### 将 config.tex 中的定理环境的对应部分替换为如下内容
% 定义单独编号,其他四个共用一个编号计数 这里只列举了五种,其他可类似定义(未定义的使用原来的也可)
\newtcbtheorem[number within=section]{defn}%
{定义}{colback=OliveGreen!10,colframe=Green!70,fonttitle=\bfseries}{def}

\newtcbtheorem[number within=section]{lemma}%
{引理}{colback=Salmon!20,colframe=Salmon!90!Black,fonttitle=\bfseries}{lem}

% 使用另一个计数器 use counter from=lemma
\newtcbtheorem[use counter from=lemma, number within=section]{them}%
{定理}{colback=SeaGreen!10!CornflowerBlue!10,colframe=RoyalPurple!55!Aquamarine!100!,fonttitle=\bfseries}{them}

\newtcbtheorem[use counter from=lemma, number within=section]{criterion}%
{准则}{colback=green!5,colframe=green!35!black,fonttitle=\bfseries}{cri}

\newtcbtheorem[use counter from=lemma, number within=section]{corollary}%
{推论}{colback=Emerald!10,colframe=cyan!40!black,fonttitle=\bfseries}{cor}
% colback=red!5,colframe=red!75!black

% 这个颜色我不喜欢
%\newtcbtheorem[number within=section]{proposition}%
%{命题}{colback=red!5,colframe=red!75!black,fonttitle=\bfseries}{cor}

% .... 命题 例 注 证明 解 使用之前的就可以(全文都是这种框框就很丑了),也可以按照上述定义 ...
\renewenvironment{proof}{\par\textbf{证明:}\;}{\qed\par}
\newenvironment{solution}{\par{\textbf{解:}}\;}{\qed\par}
\newtheorem{proposition}{\indent 命题}[section]
\newtheorem{example}{\indent \color{SeaGreen}{例}}[section] % 绿色文字的 例 ,不需要就去除\color{SeaGreen}{}
\newtheorem*{rmk}{\indent 注}
\usepackage{amssymb}
\setmathfont{LatinModernMath-Regular}
\setmathfont[range=\mathbb]{TeXGyrePagellaMath-Regular}
\def\d{\mathrm{d}}
\def\R{\mathbb{R}}
\def\C{\mathbb{C}}
\def\Q{\mathbb{Q}}
\def\N{\mathbb{N}}
\def\Z{\mathbb{Z}}
\newcommand{\bs}[1]{\boldsymbol{#1}}
\newcommand{\ora}[1]{\overrightarrow{#1}}
\newcommand{\myspace}[1]{\par\vspace{#1\baselineskip}}
\newcommand{\xrowht}[2][0]{\addstackgap[.5\dimexpr#2\relax]{\vphantom{#1}}}
\newenvironment{ca}[1][1]{\linespread{#1} \selectfont \begin{cases}}{\end{cases}}
\newenvironment{vx}[1][1]{\linespread{#1} \selectfont \begin{vmatrix}}{\end{vmatrix}}
\newcommand{\tabincell}[2]{\begin{tabular}{@{}#1@{}}#2\end{tabular}}
\newcommand{\pll}{\kern 0.56em/\kern -0.8em /\kern 0.56em}
\newcommand{\dive}[1][F]{\mathrm{div}\;\bs{#1}}
\newcommand{\rotn}[1][A]{\mathrm{rot}\;\bs{#1}}
\usepackage{xeCJK}
\setCJKmainfont{SimSun}[BoldFont={SimHei}, ItalicFont={KaiTi}] % 设置中文支持

\newcommand{\point}[1]{\item {#1}}
\newenvironment{para}[1]{%
\ifcase#1\relax
\begin{enumerate}[label=\arabic*.] % 1.2.3.
\or
\begin{enumerate}[label=\textcircled{\arabic*}] % ①②③
\or
\begin{enumerate}[label=(\roman*)] % (i)(ii)(iii)
\else
\begin{enumerate}[label=\arabic*.] % 默认格式
\fi
}{
\end{enumerate}
}

\def\myIndex{0}
% \input{\path/cover_package_\myIndex.tex}

\def\myTitle{抽象代数笔记}
\def\myAuthor{Zhang Liang}
\def\myDateCover{\today}
\def\myDateForeword{\today}
\def\myForeword{前言标题}
\def\myForewordText{
    前言内容
}
\def\mySubheading{副标题}


\begin{document}
	% \input{\path/cover_text_\myIndex.tex}

\newpage
\thispagestyle{empty}
\begin{center}
    \Huge\textbf{\myForeword}
\end{center}
\myForewordText
\begin{flushright}
    \begin{tabular}{c}
        \myDateForeword
    \end{tabular}
\end{flushright}

\newpage
\pagestyle{plain}
\setcounter{page}{1}
\pagenumbering{Roman}
\tableofcontents

\newpage
\pagenumbering{arabic}
\setcounter{chapter}{0}
\setcounter{page}{0}

\pagestyle{fancy}
\fancyfoot[C]{\thepage}
\renewcommand{\headrulewidth}{0.4pt}
\renewcommand{\footrulewidth}{0pt}








	\else
	\fi
	%标题
	\chapter{附录}
	这一部分中,对于正文中因为逻辑结构无法提及的部分,进行补充。包括特殊函数,有趣的数学概念,一些命题的全新解法,以及难以推导的公式
	证明可能使用复分析、实分析、泛函等超纲内容
	%--------------------正文---------------------------
	
	%附录:不定积分初等性判定
	\section{原函数初等性的判定方法}
		\subsection{切比雪夫定理}
			\begin{them}{切比雪夫定理}{}
				设$m,n,p\in \Q-\{0\}$,那么以下积分
				\begin{equation}
					\int x^m(a+bx^n)^p \d x
				\end{equation}
				初等的充要条件是:$p,\frac{m+1}{n},\frac{m+1}{n}+p$中至少有一个为整数
			\end{them}
		\subsection{刘维尔定理}
			在介绍刘维尔定理前,需要先介绍一些微分代数的概念:
			
			首先我们扩展微分的概念。我们将满足类似乘法、除法微分性质的泛函也称为微分。
			
			先引入微分域及其常数域
			\begin{para}{0}
				\point{微分域}
					\begin{defn}{微分域}{}
						一个由函数组成的域$F$及其上的一个算子$\delta:F \rightarrow F$,如果$\forall f,g \in F$有:
						
						$\ding{172} \delta(f+g) = \delta(f)+\delta(g)$
						
						$\ding{173} \delta(fg) = \delta(f)\cdot g+f \cdot \delta(g)$
						
						那么称$(F,\delta)$是一个微分域
					\end{defn}
					容易验证$\delta$是线性算子,于是我们有时简记$\delta(f)$为$\delta f$
					\begin{defn}{微分域的常数域}{}
						微分域$(F,\delta)$的常数域定义为:
						
						$Con (F,\delta) =\{f \in F| \delta f = 0\}$
					\end{defn}
					同时定义域的扩张:
					\begin{defn}{域的扩张}{}
						设$F,K$是两个域,并且$K$是满足$F \subseteq K$且包含$h \subseteq K$的最小域(即$K$是任何满足上述条件的域的子域),记作$K = F(h)$
					\end{defn}
					作为接下来内容的预备,我们先验证那些显然的微分性质:
					\begin{proposition}
						$\delta C = 0$,其中$C$为常数 
					\end{proposition}
					\begin{proof}
						只需要验证$\delta 1 = 0$
						
						那么有:$\delta (1\cdot 1) = \delta 1 \cdot 1 + 1 \cdot \delta 1=2\delta 1$
						
						于是有$\delta 1 = 0$,利用微分的线性即得证。
					\end{proof}
					\begin{proposition}
						$\delta \left(\frac{f}{g}\right)=\frac{\delta f \cdot g - f \delta g}{g^2}$
					\end{proposition}
					\begin{proof}
						首先推导$\delta \left(\frac{1}{g}\right)$
						
						$\because \delta 1 = \delta \left(g \cdot \frac{1}{g}\right) = 0$
						
						$\Rightarrow \delta g \frac{1}{g}+g \delta \left(\frac{1}{g}\right)=0$
						
						$\Rightarrow \delta \left(\frac{1}{g}\right) = -\frac{\delta g}{g^2}$
						
						于是$\delta \left(\frac{f}{g}\right) = \delta \left(f \cdot \frac{1}{g}\right)$
						
						$=\delta f \frac{1}{g}-f \frac{\delta g}{g^2} = \frac{\delta f \cdot g - f \delta g}{g^2}$
					\end{proof}
				\point{微分域的初等扩张}
					接下来讨论什么是“初等”的函数。
					\begin{defn}{微分域的初等扩张}{}
						设$(F,\delta),(K,\delta)$是两个微分域,$h \in K$并且$K = F(h)$,那么:
						
						$\ding{172}$如果存在$F$中的一个多项式$p(x) \in F[x]$,有$p(h)=0$,那么称$h$是$F$的一个代数元素,$K=F(h)$是$F$的单代数扩张
						
						$\ding{173}$如果存在$F$中的一个函数$f$,使得$\delta h = \frac{\delta f}{f}$,那么称$K=F(h)$是$F$的单对数扩张
						
						$\ding{173}$如果存在$F$中的一个函数$f$,使得$\frac{\delta h}{h} = \delta f$,那么称$K=F(h)$是$F$的单指数扩张。
						
						单对数扩张和单指数扩张统称为单超越扩张,其对应的$h$称为$F$的超越元素;以上三种扩张统称为单初等扩张
						
						有限次初等扩张的复合称为初等扩张
					\end{defn}
					我们也可以在此以另外的方式定义出初等函数:
					\begin{defn}{初等函数}{}
						如果函数$f$处于微分域$\left(C(x),\frac{\d}{\d x}\right)$的某个初等扩张中,那么称$f$是一个初等函数
					\end{defn}
					接下来就可以给出刘维尔定理了。
				\point{刘维尔定理}
					\begin{them}{刘维尔定理}
						设$(F,\delta),(K,\delta)$是两个微分域,$K$是$F$的初等扩张,并且$Con(F,\delta)=Con(K,\delta)$,且$\forall f \in F,\exists g \in K,s.t.\delta g = f$
						
						那么一定$\exists c_1,\cdots,c_n \in Con(F,\delta),u_1,\cdots,u_n,v \in F$,使得
						\begin{equation}
							g = \sum\limits_{i=1}^{n} c_i \ln (u_i)+v
						\end{equation}
					\end{them}
			\end{para}
			
			
	%附录:超越积分的特殊解法
	\section{一些超越积分的特殊解法}
		\subsection{Direchlet积分}
		
	
	\ifx\allfiles\undefined
\end{document}
\fi