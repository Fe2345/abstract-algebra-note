\ifx\allfiles\undefined
\documentclass[12pt, a4paper, oneside, UTF8]{ctexbook}
\setCJKmainfont{SimSun}
\def\path{../config}
\input{../config/_config}
\begin{document}
	% \input{../config/cover}
	\else
	\fi
	%标题
	\chapter{附录}
	这一部分中,对于正文中因为逻辑结构无法提及的部分,进行补充。
	%--------------------正文---------------------------
		\section{一些典型的域}
			\subsection{$F_p$}
				首先约定,这一部分的讨论中,都认为$p$是一个素数。
				
				我们首先讨论的是一个典型的有限域——模$p$剩余类域。
				
				首先给出定义。
				\begin{defn}{$n_F$}{}
					设$F$是一个域,定义以下映射$N:\N \ni n \mapsto n_F \in F$,满足:
					
					$N(0)=0_F,N(n+1)=n_F+1_F$
					
					如此定义的$n_F$称为$F$中的$n$
				\end{defn}
				\begin{defn}{$F_p$}{}
					设$F$是一个域,$Char F \geqslant p$且$p$是一个素数,
					
					我们定义$F_p = \{0_F,\cdots,p_F\}$
					
					并定义其中的加法和乘法为:
					
					$a_F +_F b_F := (a_F + b_F) mod p_F$
					
					$a_F {\cdot}_F b_F := (a_F \cdot b_F) mod p_F$
				\end{defn}
				这里的模运算定义为$a_F mod b_F = (a mod b)_F$。
				\begin{them}{$F_p$没有真子域}{}
					设$p$是一个素数,那么域$F_p$不存在真子域,即$F_p \textbackslash E \rightarrow E = F_p$
				\end{them}
				\begin{proof}
					
				\end{proof}
			\subsection{$\Q$}
				\begin{them}{$Q$没有真子域}
					aaa
				\end{them}
	
	\ifx\allfiles\undefined
\end{document}
\fi