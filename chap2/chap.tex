\ifx\allfiles\undefined
\documentclass[12pt, a4paper, oneside, UTF8]{ctexbook}
\def\path{../config}
\input{../config/_config}
\begin{document}
% \input{../config/cover}
\else
\fi
%标题
\chapter{环、模}
	\section{环的定义}
		\subsection{环的定义}
			\begin{defn}{}
				设$R$是一个集合,如果存在两个运算$+ : R \times R \rightarrow R$和$\cdot : R \times R \rightarrow R$

				分别称为加法和乘法,满足下列条件:

				\ding{172} (加法单位元存在)存在一个元素$0_R \in R$,称为加法单位元,使得对于任意$x \in R$,有$x + 0_R = 0_R + x = x$。

				\ding{173} (加法交换律)$\forall x, y \in R,x + y = y + x$

				\ding{174} (加法结合律)$\forall x, y, z \in R,(x + y) + z = x + (y + z)$

				\ding{175} (加法逆存在)$\forall x \in R,\exists -x \in R$,称为加法逆,使得$x + (-x) = 0_R$

				\ding{176} (乘法结合律)$\forall x, y, z \in R,(x \cdot y) \cdot z = x \cdot (y \cdot z)$

				\ding{177} (左分配律)$\forall x, y, z \in R,x \cdot (y + z) = x \cdot y + x \cdot z$
							
							(右分配律)$\forall x, y, z \in R,(y + z) \cdot x = y \cdot x + z \cdot x$

				那么我们称$(R,+,\cdot)$是一个环,简称为环$R$。
			\end{defn}
			相比域的定义,环的定义仅涉及6条性质,去除了单位元存在、可交换、可逆三条性质。

			在研究环时,我们有时也会考虑存在单位元和可交换的环,因此有以下定义:
			\begin{defn}{交换环、幺环}{}
				如果环$R$满足:$\forall x,y \in R,x \cdot y = y \cdot x$,那么我们称$R$是一个交换环;

				如果环$R$满足:$\exists 1_R \in R,\forall x \in R,1_R \cdot x = x \cdot 1_R = x$,称为乘法单位元,
			\end{defn}
		\subsection{环的性质}
		\subsection{子环}
			我们也类似地提出后续我们会提及的子环的概念。
			\begin{defn}{子环}{}
				设$R$是一个环,集合$S \subseteq R$,
				
				如果$S$对$R$上的加法和乘法也构成一个环,那么我们称$S$是$R$的子环。
			\end{defn}
	\section{环的同态}
		我们类似于域的同态,定义出环的同态。
		\subsection{定义}
			\begin{defn}{环同态}{}
				设$R,S$是两个环,如果映射$\varphi : R \rightarrow S$满足:

				$\forall a,b \in R,\varphi (a+b)=\varphi (a)+\varphi (b),\varphi (ab)=\varphi (a)\varphi (b)$

				并且如果$R,S$均是幺环,$\varphi (1_R)=1_S$
				
				那么我们称$\varphi $是一个$R$到$S$的环同态。
			\end{defn}
			\begin{defn}{单同态、满同态、同构}{}
				假设有环同态$\psi :R \rightarrow S$

				如果$\psi $是单射,那么称它是一个单同态;

				如果$\psi $是满射,那么称它是一个满同态;

				如果$\psi $是双射,称它是一个同构;
			\end{defn}
			同构是最严格的同态,表示两个环在结构上是完全相同的,有以下显而易见的事实:
			\begin{proposition}
				如果$\psi : R \rightarrow S$是一个环同构,那么$\psi^{-1} : S \rightarrow R$也是一个环同构
			\end{proposition}
			\begin{proof}
				$\psi \left(\psi^{-1} (a)+\psi^{-1} (b)\right)=\psi \left(\psi (a)\right)+\psi \left(\psi (b)\right)=a+b$

				因为$\psi $是双射,所以有$\psi^{-1} (a)+\psi^{-1} (b)=\psi (a+b)$

				同理,$\psi \left(\psi^{-1} (a)\psi^{-1} (b)\right)=\psi \left(\psi (a)\right)\psi \left(\psi (b)\right)=ab$

				$\Rightarrow \psi^{-1} (a)\psi^{-1} (b)=\psi (ab)$,于是命题得证
			\end{proof}
		\subsection{同态的核、像}
			\begin{defn}{环同态的核、像}{}
				设$\psi : R \rightarrow S$是一个环同态,我们定义:

				$\ker \psi = \{a \in R | \psi (a) = 0_S\}$,称为$\psi $的核

				$\text{Im } \psi = \{\psi (a) | a \in R\}$,称为$\psi $的像
			\end{defn}
			与域的同态不同,环的同态的核并不是平凡的,因为域同态未必是单射。因此,我们需要研究环同态的核与像。

			但是,受限于目前的知识,我们暂时无法证明核与像的一些进阶性质,我们仅仅证明一些简单的性质。
			\begin{proposition}
				设$\psi :R \rightarrow S$是一个环同态,那么$\ker \psi $是一个$R$的子环
			\end{proposition}
			\begin{proof}
				
			\end{proof}
	\section{环的理想}
		\begin{defn}{理想}{}
			设$R$是一个环,$R$的子环$R$如果满足:

			$\forall a \in R,b \in S,ab \in S$,那么我们称$S$是$R$的一个理想
		\end{defn}
		我们观察到:一些环,比如说$\Z$,他们有一种特殊的理想,比如说$\forall m \in \Z,m\Z$是$\Z$的一个理想。我们把这种直接由一个元素“生成”的理想叫主理想。
		\begin{defn}{主理想}{}
			设$R$是一个环,如果$R$的一个理想$S$满足:

			$\exists a \in R,S=aR := \{ab | b \in R\}$,那么我们称$S$是由$a$生成的主理想,记作$(a)$
		\end{defn}
	\section{模}
		本节我们研究一种“环上的线性空间”,也就是模
		\subsection{模的定义}
			\begin{defn}{左模}{}
				设$R$是一个环,$M$是一个集合,如果存在两个运算$+ : M \times M \rightarrow M,\cdot_R : R \times M \rightarrow M$
				
				分别称为称为加法和纯量乘法,满足下列条件:

				\ding{172} $\exists 0_M \in M$,称为加法单位元,$\forall \alpha \in M,0_M + \alpha =\alpha +0_M=\alpha $

				\ding{173} $\forall \alpha,\beta  \in M,\alpha +\beta =\beta +\alpha $

				\ding{174} $\forall \alpha,\beta,\gamma  \in M,(\alpha +\beta )+\gamma =\alpha +(\beta +\gamma )$

				\ding{175} $\forall \alpha \in M,\exists -\alpha \in M$,称为加法逆,使得$\alpha +(-\alpha )=0_M$

				\ding{176} $\forall a,b \in R,\alpha \in M,a(b\alpha )=(ab)\alpha$

				\ding{177} $\forall a \in R,\alpha ,\beta \in M,a(\alpha +\beta )=a\alpha +a\beta $

				\ding{178} $\forall a,b \in R,\alpha \in M,(a+b)\alpha =a\alpha +b\alpha $

				那么我们称$M$是一个左$R-$模
			\end{defn}
			类似地,我们也有右模的定义:
			\begin{defn}{右模}{}
				设$R$是一个环,$M$是一个集合,如果存在两个运算$+ : M \times M \rightarrow M,\cdot_R : M \times R \rightarrow M$
				
				分别称为称为加法和纯量乘法,满足下列条件:

				\ding{172} $\exists 0_M \in M$,称为加法单位元,$\forall \alpha \in M,0_M + \alpha =\alpha +0_M=\alpha $

				\ding{173} $\forall \alpha,\beta  \in M,\alpha +\beta =\beta +\alpha $

				\ding{174} $\forall \alpha,\beta,\gamma  \in M,(\alpha +\beta )+\gamma =\alpha +(\beta +\gamma )$

				\ding{175} $\forall \alpha \in M,\exists -\alpha \in M$,称为加法逆,使得$\alpha +(-\alpha )=0_M$

				\ding{176} $\forall a,b \in R,\alpha \in M,(\alpha a)b=\alpha (ab)$

				\ding{177} $\forall a \in R,\alpha ,\beta \in M,(\alpha +\beta )a=\alpha a+\beta b$

				\ding{178} $\forall a,b \in R,\alpha \in M,\alpha (a+b)=\alpha a+\alpha b$

				那么我们称$M$是一个右$R-$模
			\end{defn}
			如果$M$兼具左模和右模的特征,我们称$M$是一个双模:
			\begin{defn}{双模}{}
				如果$M$既是左$R-$模,又是右$S-$模,并且满足:

				$(a \alpha )b=a(\alpha b)$

				那么我们称$M$是一个$(R,S)-$双模
			\end{defn}
			我们也知道,环不一定有乘法单位元,因此有以下定义:
			\begin{defn}{幺模}{}
				设$R$是幺环,$M$是一个$R-$模

				如果$\forall \alpha \in M,1_R \cdot \alpha =\alpha $,那么我们称$M$是一个幺模。
			\end{defn}
			我们之所以刻意区分左右模,并且还提出双模的概念,是因为环不一定交换,因此$a\alpha $和$\alpha a$并不等价。

			以下命题指出了一个事实:只有在环是交换幺环时,左右幺模才在交换乘法次序的关系下等价。
			\begin{proposition}
				设$R$是一个交换环,$M$既是左$R-$模,又是右$R-$模,那么

				$E = \{(a\alpha ,\alpha a)| a \in R,\alpha \in M\} \subseteq M \times M$是一个等价关系
			\end{proposition}
			\begin{proof}
				
			\end{proof}
			在后续中,除非特别做区分,我们都假定我们说的模指的是左模。
		\subsection{模的同态}
			\begin{defn}{模同态}{}
				设$M,N$是两个$R-$模,如果映射$\psi  : M \rightarrow N$满足:

				$\forall \alpha ,\beta  \in M,\psi (\alpha +\beta )=\psi (\alpha )+\psi (\beta )$

				$\forall a \in R,\alpha  \in M,\psi (a\alpha )=a\psi (\alpha )$

				那么称$\psi $是一个从$M$到$N$的模同态。
			\end{defn}
\ifx\allfiles\undefined
\end{document}
\fi