\ifx\allfiles\undefined
\documentclass[12pt, a4paper, oneside, UTF8]{ctexbook}
\def\path{../config}
\input{../config/_config}
\begin{document}
% \input{../config/cover}
\else
\fi
%标题
\chapter{环、模}
	\section{环的定义}
		\begin{defn}{}
			设$R$是一个集合,如果存在两个运算$+ : R \times R \rightarrow R$和$\cdot : R \times R \rightarrow R$

			分别称为加法和乘法,满足下列条件:

			\ding{172} (加法单位元存在)存在一个元素$0_R \in R$,使得对于任意$x \in R$,有$x + 0_R = 0_R + x = x$。

			\ding{173} (加法交换律)$\forall x, y \in R,x + y = y + x$

			\ding{174} (加法结合律)$\forall x, y, z \in R,(x + y) + z = x + (y + z)$

			\ding{175} (加法逆存在)$\forall x \in R,\exists -x \in R$,称为加法逆,使得$x + (-x) = 0_R$

			\ding{176} (乘法结合律)$\forall x, y, z \in R,(x \cdot y) \cdot z = x \cdot (y \cdot z)$

			\ding{177} (左分配律)$\forall x, y, z \in R,x \cdot (y + z) = x \cdot y + x \cdot z$
						
						(右分配律)$\forall x, y, z \in R,(y + z) \cdot x = y \cdot x + z \cdot x$

			那么我们称$(R,+,\cdot)$是一个环,简称为环$R$。
		\end{defn}
		\begin{defn}{交换环、幺环}{}
			
		\end{defn}
\ifx\allfiles\undefined
\end{document}
\fi