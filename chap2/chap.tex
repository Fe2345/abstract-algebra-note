\ifx\allfiles\undefined
\documentclass[12pt, a4paper, oneside, UTF8]{ctexbook}
\def\path{../config}
\input{../config/_config}
\begin{document}
% \input{../config/cover}
\else
\fi
%标题
\chapter{环、模}
	\section{环的定义}
		\subsection{环的定义}
			\begin{defn}{}
				设$R$是一个集合,如果存在两个运算$+ : R \times R \rightarrow R$和$\cdot : R \times R \rightarrow R$

				分别称为加法和乘法,满足下列条件:

				\ding{172} (加法单位元存在)存在一个元素$0_R \in R$,称为加法单位元,使得对于任意$x \in R$,有$x + 0_R = 0_R + x = x$。

				\ding{173} (加法交换律)$\forall x, y \in R,x + y = y + x$

				\ding{174} (加法结合律)$\forall x, y, z \in R,(x + y) + z = x + (y + z)$

				\ding{175} (加法逆存在)$\forall x \in R,\exists -x \in R$,称为加法逆,使得$x + (-x) = 0_R$

				\ding{176} (乘法结合律)$\forall x, y, z \in R,(x \cdot y) \cdot z = x \cdot (y \cdot z)$

				\ding{177} (左分配律)$\forall x, y, z \in R,x \cdot (y + z) = x \cdot y + x \cdot z$
							
							(右分配律)$\forall x, y, z \in R,(y + z) \cdot x = y \cdot x + z \cdot x$

				那么我们称$(R,+,\cdot)$是一个环,简称为环$R$。
			\end{defn}
			相比域的定义,环的定义仅涉及6条性质,去除了单位元存在、可交换、可逆三条性质。

			在研究环时,我们有时也会考虑存在单位元和可交换的环,因此有以下定义:
			\begin{defn}{交换环、幺环}{}
				如果环$R$满足:$\forall x,y \in R,x \cdot y = y \cdot x$,那么我们称$R$是一个交换环;

				如果环$R$满足:$\exists 1_R \in R,\forall x \in R,1_R \cdot x = x \cdot 1_R = x$,称为乘法单位元,
			\end{defn}
		\subsection{环的性质}
		\subsection{整环}
			\begin{defn}{零因子}{}
				设$R$是一个环,如果$\exists x,y \in R,x,y \neq 0_R$,使得$x \cdot y = 0_R$,那么我们称$x,y$是$R$的零因子。
			\end{defn}
			\begin{defn}{整环}{}
				如果环$R$是一个交换幺环,并且不包含零因子,那么我们称$R$是一个整环。
			\end{defn}
			\begin{them}{循环的整环必有素零因子}{}
				设$R$是一个整环,
				
				我们定义:$N : \Z \ni n \mapsto n_R \in R$,满足$(n+1)_F = n_F+1_F$

				如果$\exists a \in R,a \neq 0,\exists n \in \N_+,n_F a = 0_R$

				那么存在素数$p$,$\forall b \in R,p_R b=0_R$
			\end{them}
			\begin{proof}
				取$\forall b \in R$,

				$0_R=0_R\cdot b=(n_R a)b=a(n_R b)$

				因为$a \neq 0$,而$R$是整环,所以一定有$n_R b=0$,因此,$\{k\in \N_+|k_R b=0\}$不是空集。

				取$p$为使$p_R b=0$的最小正整数。如果$p$是素数,命题成立;

				如果$p$不是素数,那么只需要对$p$作唯一分解,那么有$\left(\prod\limits_{i=1}^{q} {p_i}_R\right)b=0$

				那么,一定存在一个${p_i}_R b=0$,此时命题也是成立的。
			\end{proof}
		\subsection{子环}
			我们也类似地提出后续我们会提及的子环的概念。
			\begin{defn}{子环}{}
				设$R$是一个环,集合$S \subseteq R$,
				
				如果$S$对$R$上的加法和乘法也构成一个环,那么我们称$S$是$R$的子环。
			\end{defn}
	\section{环的同态}
		我们类似于域的同态,定义出环的同态。
		\subsection{定义}
			\begin{defn}{环同态}{}
				设$R,S$是两个环,如果映射$\varphi : R \rightarrow S$满足:

				$\forall a,b \in R,\varphi (a+b)=\varphi (a)+\varphi (b),\varphi (ab)=\varphi (a)\varphi (b)$

				并且如果$R,S$均是幺环,$\varphi (1_R)=1_S$
				
				那么我们称$\varphi $是一个$R$到$S$的环同态。
			\end{defn}
			显然,同态一定将零元映射到零元
			\begin{proposition}
				设$\varphi : R \rightarrow S$是一个环同态,那么$\varphi (0_R) = 0_S$
			\end{proposition}
			\begin{proof}
				$\varphi (0_R) = \varphi (0_R + 0_R) = \varphi (0_R) + \varphi (0_R)$

				$\Rightarrow -\varphi (0_R)+\varphi (0_R) = -\varphi (0_R)+\varphi (0_R)+\varphi (0_R)$

				$\Rightarrow \varphi (0_R)=0_S$
			\end{proof}
			\begin{defn}{单同态、满同态、同构}{}
				假设有环同态$\psi :R \rightarrow S$

				如果$\psi $是单射,那么称它是一个单同态;

				如果$\psi $是满射,那么称它是一个满同态;

				如果$\psi $是双射,称它是一个同构;
			\end{defn}
			同构是最严格的同态,表示两个环在结构上是完全相同的,有以下显而易见的事实:
			\begin{proposition}
				如果$\psi : R \rightarrow S$是一个环同构,那么$\psi^{-1} : S \rightarrow R$也是一个环同构
			\end{proposition}
			\begin{proof}
				$\psi \left(\psi^{-1} (a)+\psi^{-1} (b)\right)=\psi \left(\psi (a)\right)+\psi \left(\psi (b)\right)=a+b$

				因为$\psi $是双射,所以有$\psi^{-1} (a)+\psi^{-1} (b)=\psi (a+b)$

				同理,$\psi \left(\psi^{-1} (a)\psi^{-1} (b)\right)=\psi \left(\psi (a)\right)\psi \left(\psi (b)\right)=ab$

				$\Rightarrow \psi^{-1} (a)\psi^{-1} (b)=\psi (ab)$,于是命题得证
			\end{proof}
		\subsection{同态的核、像}
			\begin{defn}{环同态的核、像}{}
				设$\psi : R \rightarrow S$是一个环同态,我们定义:

				$\ker \psi = \{a \in R | \psi (a) = 0_S\}$,称为$\psi $的核

				$\text{Im } \psi = \{\psi (a) | a \in R\}$,称为$\psi $的像
			\end{defn}
			与域的同态不同,环的同态的核并不是平凡的,因为域同态未必是单射。因此,我们需要研究环同态的核与像。

			但是,受限于目前的知识,我们暂时无法证明核与像的一些进阶性质,我们仅仅证明一些简单的性质。
			\begin{proposition}
				设$\psi :R \rightarrow S$是一个环同态,那么$\ker \psi $是一个$R$的子环
			\end{proposition}
			\begin{proof}
				取$\forall a,b \in \ker \psi$,那么有$\psi (a)=\psi (b)=0$

				我们注意到:$\psi (0_R) = 0_S \Rightarrow 0_R \in \ker \psi ,a +0_R=a$

				$\psi \left(a+(-a)\right)=0_S \Rightarrow \psi (a)+\psi (-a)=0_S \Rightarrow \psi (-a)=0_S \Rightarrow -a \in \ker \psi,a+(-a)=0_R$

				加法的交换律、结合律,乘法的结合律,左、右分配律是显然成立的。
			\end{proof}
			\begin{proposition}
				设$\psi :R \rightarrow S$是一个环同态,那么$\text{Im } \psi $是一个$S$的子环
			\end{proposition}
			\begin{proof}
				取$\forall a,b \in \text{Im } \psi$,那么有$\exists x,y \in R,\psi (x)=a,\psi (y)=b$

				我们注意到:$\psi (0_R) = 0_S \Rightarrow 0_S \in \text{Im } \psi ,\psi (x)+\psi (0_R)=\psi (x)$

				$\psi (x)+\psi (-x)=\psi (x+(-x))=\psi (0_R)=0_S \Rightarrow \psi (-x)=-\psi (x)=-a \in \text{Im } \psi ,a+(-a)=0_S$

				加法的交换律、结合律,乘法的结合律,左、右分配律是显然成立的。
			\end{proof}
	\section{环的理想}
		\subsection{理想的定义}
			\begin{defn}{理想}{}
				设$R$是一个环,$R$是$R$的一个子环。

				如果$\forall a \in S,b \in R,ab \in S$,那么我们称$S$是$R$的一个左理想;

				如果$\forall a \in S,b \in R,ba \in S$,那么我们称$S$是$R$的一个右理想;

				如果$S$既是$R$的左理想,又是$R$的右理想,那么我们称$S$是$R$的一个双理想。
			\end{defn}
			显然,$\{0_R\},R$都是$R$的理想,我们称之为平凡理想。

			我们观察到:一些环,比如说$\Z$,他们有一种特殊的理想,比如说$\forall m \in \Z,m\Z$是$\Z$的一个理想。我们把这种直接由一个元素“生成”的理想叫主理想。
			\begin{defn}{主理想}{}
				设$R$是一个环,如果$R$的一个理想$S$满足:

				$\exists a \in R,S=aR := \{ab | b \in R\}$,那么我们称$S$是由$a$生成的主理想,记作$(a)$
			\end{defn}
		\subsection{理想的性质}
		\begin{para}{0}
			\point{}
				\begin{proposition}
					设$\psi :R \rightarrow S$是一个环同态,那么$\ker \psi$是$S$的一个理想
				\end{proposition}
				\begin{proof}
					取$\forall a \in \ker \psi$,依核的定义,$\psi (a)=0_S$

					那么,$\forall b \in R,\psi (ab)=\psi (a)\psi (b)=0_S,\psi (ba)=\psi (b)\psi (a)=0_S$

					因此,$ab,ba \in \ker \psi$,于是命题得证。
				\end{proof}
			\point{}
				\begin{proposition}
					设$R$是一个幺环,$S$是$R$的一个双理想,如果$1_R \in S$,那么$S=R$
				\end{proposition}
				\begin{proof}
					依理想的定义,$\forall b \in R,1_R b=b 1_R = b \in S$,所以$R \subseteq S$,所以必须有$R=S$
				\end{proof}
				我们可以立即得出,最特殊的幺环——域,也可以应用上述性质
				\begin{corollary}{域没有非平凡理想}{}
					域只有平凡理想
				\end{corollary}
				\begin{proof}
					设$F$是一个域,$S$是$F$的一个理想

					因为域中任意元素都有逆元素,所以$\forall a \in S,a^{-1} \in F$

					而按照理想的概念,$a \cdot a^{-1}=1_F \in S$

					而按照前面的命题,$1_F \in S$,那么一定有$S = F$,它是一个平凡理想;
				\end{proof}
		\end{para}
	\section{商环}
		\subsection{商环的定义}
			我们首先定义等价类,随后定义商环
			\begin{defn}{关于环的理想的等价类}{}
				设$R$是一个环,$S$是$R$的一个理想
				
				我们定义$R \times R$上的一个等价关系:$\underset{I}{\sim} : \{(x,y) | x-y \in I\} \subseteq R \times R$

				并定义$a \in R$关于$\underset{I}{\sim}$的等价类为:$a +I :=\overline{a} := \{x \in R | x \underset{I}{\sim} a\}$
			\end{defn}
			我们其实还需要验证以上关系的确是一个等价关系:

			首先,$\forall a \in R,a \sim a$,因为$a-a=0_R \in I$,这说明自反性成立;

			其次,如果$a \sim b$,那么有$a-b \in I$,而$I$是一个理想,所以一定有$b-a \in I$,所以$b \sim a$,这说明对称性成立;

			最后,如果$a \sim b,b \sim c$,那么有$a-b \in I,b-c \in I$,而$I$是一个理想,所以一定有$(a-b)+(b-c)=(a-c) \in I$,所以$a \sim c$,这说明传递性成立;
			\begin{defn}{商环}{}
				设$R$是一个环,$I$是$R$的一个理想,那么我们定义:

				$R/I = \{a +I | a \in R\}$,称为$R$关于$I$的商环

				并定义其中的环加法和环乘法为:

				$(a+I)+(b+I)=(a+b)+I,(a+I)\cdot(b+I)=(a\cdot b)+I$
			\end{defn}
			事实上,由于$a+I$是一个等价类,我们还需要验证,如果$a_1+I=a_2+I$,即同一等价类选取不同单位元下,运算结果是一致的。
			\begin{proposition}{商环的加法和乘法是良定义的}{}
				设$R$是一个环,$I$是$R$的一个理想,那么如果$a_1+I=a_2+I,b_1+I=b_2+I$

				那么$(a_1+I)+(b_1+I)=(a_2+I)+(b_2+I),(a_1+I)\cdot (b_1+I)=(a_2+I)\cdot (b_2+I)$
			\end{proposition}
			\begin{proof}
				对于第一条,只需证明$(a_1+b_1)-(a_2+b_2) \in I$。

				因为$(a_1+b_1)-(a_2+b_2)=(a_1-a_2)+(b_1-b_2)$,而$a_1-a_2 \in I,b_1-b_2 \in I$,所以$(a_1+b_1)-(a_2+b_2) \in I$
				
				对于第二条,只需证明$(a_1\cdot b_1)-(a_2\cdot b_2) \in I$。

				因为$(a_1\cdot b_1)-(a_2\cdot b_2)=(a_1\cdot b_1-a_1\cdot b_2)+(a_1\cdot b_2-a_2\cdot b_2)$

				$=a_1(b_1-b_2)+(a_1-a_2)b_2 \in I$,因为$I$是一个理想。
			\end{proof}
	\section{同态基本定理}
		\subsection{同态基本定理}
			\begin{them}{同态第一基本定理}{}
				设$R,S$是两个环,$\varphi : R \rightarrow S$是一个环同态,那么:

				$R / \ker \varphi \cong \text{Im }\varphi $
			\end{them}
			\begin{proof}
				我们考虑以下映射:

				$\psi : R / \ker \varphi \ni a+\ker \varphi  \mapsto \varphi (a) \in \text{Im }\varphi $

				首先,$\psi \left((a+\ker \varphi )+(b+\ker \varphi )\right)=\psi \left((a+b)+\ker \varphi \right)=\varphi (a+b)=\varphi (a)+\varphi (b)=\psi (a+\ker \varphi )+\psi (b+\ker \varphi )$
				
				$\psi \left((a+\ker \varphi )\cdot (b+\ker \varphi )\right)=\psi (ab+\ker \varphi )=\varphi (ab)=\varphi (a)\varphi (b)=\psi (a+\ker \varphi )\cdot \psi (b+\ker \varphi )$

				这说明$\psi $是一个同态,我们接下来只需要证明双射性。

				首先证明单射性,如果$\psi (a+\ker \varphi )=\psi (b+\ker \varphi )$,即$\varphi (a)=\varphi (b)$,那么$\varphi (a)-\varphi (b)=\varphi (a-b)=0_R$,所以$a-b \in \ker \varphi $,于是$a+\ker \varphi =b+\ker \varphi $,单射性成立。

				满射性是显然的,因为显然$\forall b \in \text{Im }\varphi ,\exists a \in R,\varphi (a)=b,\psi (a+\ker \varphi )=\varphi (a)=b$

				于是命题得证。
			\end{proof}
	\section{模}
		本节我们研究一种“环上的线性空间”,也就是模
		\subsection{模的定义}
			\begin{defn}{左模}{}
				设$R$是一个环,$M$是一个集合,如果存在两个运算$+ : M \times M \rightarrow M,\cdot_R : R \times M \rightarrow M$
				
				分别称为称为加法和纯量乘法,满足下列条件:

				\ding{172} $\exists 0_M \in M$,称为加法单位元,$\forall \alpha \in M,0_M + \alpha =\alpha +0_M=\alpha $

				\ding{173} $\forall \alpha,\beta  \in M,\alpha +\beta =\beta +\alpha $

				\ding{174} $\forall \alpha,\beta,\gamma  \in M,(\alpha +\beta )+\gamma =\alpha +(\beta +\gamma )$

				\ding{175} $\forall \alpha \in M,\exists -\alpha \in M$,称为加法逆,使得$\alpha +(-\alpha )=0_M$

				\ding{176} $\forall a,b \in R,\alpha \in M,a(b\alpha )=(ab)\alpha$

				\ding{177} $\forall a \in R,\alpha ,\beta \in M,a(\alpha +\beta )=a\alpha +a\beta $

				\ding{178} $\forall a,b \in R,\alpha \in M,(a+b)\alpha =a\alpha +b\alpha $

				那么我们称$M$是一个左$R-$模
			\end{defn}
			类似地,我们也有右模的定义:
			\begin{defn}{右模}{}
				设$R$是一个环,$M$是一个集合,如果存在两个运算$+ : M \times M \rightarrow M,\cdot_R : M \times R \rightarrow M$
				
				分别称为称为加法和纯量乘法,满足下列条件:

				\ding{172} $\exists 0_M \in M$,称为加法单位元,$\forall \alpha \in M,0_M + \alpha =\alpha +0_M=\alpha $

				\ding{173} $\forall \alpha,\beta  \in M,\alpha +\beta =\beta +\alpha $

				\ding{174} $\forall \alpha,\beta,\gamma  \in M,(\alpha +\beta )+\gamma =\alpha +(\beta +\gamma )$

				\ding{175} $\forall \alpha \in M,\exists -\alpha \in M$,称为加法逆,使得$\alpha +(-\alpha )=0_M$

				\ding{176} $\forall a,b \in R,\alpha \in M,(\alpha a)b=\alpha (ab)$

				\ding{177} $\forall a \in R,\alpha ,\beta \in M,(\alpha +\beta )a=\alpha a+\beta b$

				\ding{178} $\forall a,b \in R,\alpha \in M,\alpha (a+b)=\alpha a+\alpha b$

				那么我们称$M$是一个右$R-$模
			\end{defn}
			如果$M$兼具左模和右模的特征,我们称$M$是一个双模:
			\begin{defn}{双模}{}
				如果$M$既是左$R-$模,又是右$S-$模,并且满足:

				$(a \alpha )b=a(\alpha b)$

				那么我们称$M$是一个$(R,S)-$双模
			\end{defn}
			我们也知道,环不一定有乘法单位元,因此有以下定义:
			\begin{defn}{幺模}{}
				设$R$是幺环,$M$是一个$R-$模

				如果$\forall \alpha \in M,1_R \cdot \alpha =\alpha $,那么我们称$M$是一个幺模。
			\end{defn}
			在后续中,除非特别做区分,我们都假定我们说的模指的是左模。
		\subsection{模的同态}
			\begin{defn}{模同态}{}
				设$M,N$是两个$R-$模,如果映射$\psi  : M \rightarrow N$满足:

				$\forall \alpha ,\beta  \in M,\psi (\alpha +\beta )=\psi (\alpha )+\psi (\beta )$

				$\forall a \in R,\alpha  \in M,\psi (a\alpha )=a\psi (\alpha )$

				那么称$\psi $是一个从$M$到$N$的模同态。
			\end{defn}
		\subsection{商模}
			\begin{defn}{商模}{}
				设$M$是一个模,$N$是$M$的一个子模
			\end{defn}
\ifx\allfiles\undefined
\end{document}
\fi